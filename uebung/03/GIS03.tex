\documentclass{article}
\usepackage{tikz}
\usepackage{pgfplots}
\usepackage{amsmath, amsfonts, amssymb}
\usepackage{enumerate}
\usepackage{hyperref}
\usepackage{listings}
\input{title.tex}

\newcommand{\N}{\mathbb{N}}
\newcommand{\Z}{\mathbb{Z}}
\newcommand{\R}{\mathbb{R}}
\begin{document}
    \maketitle
    \section{}
    \begin{enumerate}
        \item
            \begin{align*}
                a|b\Rightarrow b&=qa+0\tag{1}\\
                b|a\Rightarrow a&=q'b+0\\
                &\overset{(1)}=q'qa\\
                \Leftrightarrow q'q&=1\\
                q=q'&=1 (q,q'\in\N)\\
                \Rightarrow
                b=1\cdot a&=a
            \end{align*}
        \item
            We know $p\nmid x\Rightarrow x=q'p+r(r<p)$
            \begin{align*}
                p|(xy)\Rightarrow
                qp&=xy\\
                &=(q'p+r)y\\
                &=q'py+ry\\
                q&=q'y+\frac{ry}{p}\\
                &\Rightarrow\frac{ry}{p}\in\N\\
                &\Rightarrow\frac{y}{p}\in\N\\
                &\Rightarrow p|y
            \end{align*}
    \end{enumerate}
    \section{}
    \begin{itemize}
        \item $m=11=1011_2$
        \item $i=\lfloor\log(m)\rfloor=3$
        \item $k=5$
    \end{itemize}
    \begin{tabular}{cccc}
        i&b(i)&h&k\\
        \hline
        3&1&1&5\\
        2&1&5&25\\
        1&0&23&13\\
        0&1&23&16\\
        -1&&11
    \end{tabular}\\
    Es ist also $11=E(5,(51,11))=5^{11}\mod 51$
    \section{}
    \begin{enumerate}
        \item
            Associativity:
            Be $x,y,z\in G\times H$.
            $G,H$ are groups, thus
            $\odot_G,\odot_H$ are associative.
            For $(G\times H,\cdot)$ to be a group,
            the following conditions must hold:
            \begin{align*}
                (x\cdot y)\cdot z
                &=((x_1,x_2)\cdot(y_1,y_2))\cdot(z_1,z_2)\\
                &=(x_1\odot_G y_1,x_2\odot_H y_2
                \cdot(z_1,z_2)\\
                &=((x_1\odot_G y_1)\odot_G z_1,
                (x_2\odot_H y_2)\odot_H z_2)\\
                &=(x_1\odot_G (y_1\odot_G z_1),
                x_2\odot_H (y_2\odot_H z_2))\\
                &=x\cdot (y\cdot z)
            \end{align*}
        \item
            Identity element:
            \begin{align*}
                \underset{1\in G\times H}\exists
                \underset{v\in G\times H}\forall
                1\cdot v = v.
            \end{align*}
            With $1=(1_G, 1_H)$
            the tuple of the identities of
            $G,H$, we have
            \begin{align*}
                1\cdot v&=(1_G\odot_G v_1, 1_H\odot_H v_2)\\
                &=(v_1, v_2)=v
            \end{align*}
        \item
            Inverse element:
            \begin{align*}
                \underset{v\in G\times H}\forall
                \underset{v^{-1}\in G\times H}\exists
                v^{-1}\cdot v = v.
            \end{align*}
            With $v^{-1}=(v_1^{-1}, v_2^{-1})$
            the tuple of the inverses of v in
            $G,H$, we have
            \begin{align*}
                v^{-1}\cdot v&=
                (v_1^{-1}\odot_G v_1, v_2^{-1}\odot_H v_2)\\
                &=(1, 1)=1
            \end{align*}
    \end{enumerate}
    Thus $(G\times H,\cdot)$ is a group.
\end{document}
