\documentclass{article}
\usepackage{tikz}
\usepackage{pgfplots}
\usepackage{amsmath, amsfonts}
\usepackage{enumerate}
\usepackage{hyperref}
\usepackage{listings}
\input{title.tex}

\newcommand{\N}{\mathbb{N}}
\newcommand{\R}{\mathbb{R}}
\begin{document}
	\maketitle
	\section{}
	\begin{enumerate}[a)]
	    \item
			$D: K\times Y\rightarrow X$
			\begin{align*}
				(k_1,A)\mapsto a,
				(k_1,B)\mapsto b,
				(k_1,C)\mapsto c,\\
				(k_2,A)\mapsto c,
				(k_2,B)\mapsto a,
				(k_2,C)\mapsto b,\\
				(k_3,A)\mapsto b,
				(k_3,B)\mapsto c,
				(k_3,C)\mapsto a\,
			\end{align*}
		\item
		    Es existiert kein deterministisches $D$, da:
			\begin{align*}
				(k_1, A)\mapsto \{a,c\},
				(k_2, B)\mapsto \{a,c\},
				(k_3, C)\mapsto \{a,c\}
			\end{align*}
		\item
    		$D: K\times Y\rightarrow X$
    		\begin{align*}
    			(k_1,A)\mapsto a,
    			(k_1,B)\mapsto b,
    			(k_1,C)\mapsto c,\\
    			(k_2,A)\mapsto b,
    			(k_2,B)\mapsto a,
    			(k_2,C)\mapsto c,\\
    			(k_3,A)\mapsto a,
    			(k_3,B)\mapsto b,
    			(k_3,C)\mapsto c\,
    		\end{align*}
    \end{enumerate}
    \section{}
    \begin{tabular}{c|ccccccccc}
        hex&48&61&6c&6c&6f&77&65&65&6e\\
        \hline
        ASCII&H&a&l&l&o&w&e&e&n
    \end{tabular}

    x ist pro zeichen berechnet mit: $x(i)=y(i)\oplus k(i)(i\leq 72)$
    \section{}
    \begin{enumerate}[a)]
        \item
            $D:X\times K\rightarrow X$ mit
            $(y,k)\mapsto y(l)y(l)\hdots y(2l-1)\oplus 1$\\
            Um zu zeigen, dass $S=(X,K,E,D)$ ein verschlüsselungs Schema ist,
            Zeigen wir $D(E(x,k),k)=x$ für beliebige
            $(x,k)\in X\times K$
            \begin{align*}
                E(x,k)=(r\oplus k)||\left(x\oplus 1^l\right)
            \end{align*}
            mit $|r\oplus k|=l$, somit
            \begin{equation*}
                D(E(x,k),k)=(x\oplus 1^l)\oplus 1^l=x
            \end{equation*}
        \item
            \begin{lstlisting}[mathescape=true]
                choose $z_0$, $z_1$ from $(\{0,1\}^{128})^*$
                    where l := $|z_0|=|z_1|$;
                b := send($z_0$, $z_1$);
                // decryption algorithm
                d := b(l)b(l+1)...b(2l-1)$\oplus$ 1;
                if d = $z_0$
                    return 0;
                else
                    return 1;
            \end{lstlisting}
    \end{enumerate}
    \section{}
    Zunächst verschlüsseln wir einen text $x$.
    Sei $y:=E(x,k)$.

    Nun schicken wir 2 texte $z_0=x,z_1$.
    Seien $x_0,\hdots,x_n$, $y_0,\hdots,y_{n+1}$, sowie$b_0,\hdots,b_{n+1}$
    die Blöcke der länge 128.

    Wir wissen $r=y_0,r'=b_0$ sind die zufällig gewählten zahlen für $y,b$
    der erste verschlüsselte block von $x$,
    $y_1=(r+1\mod 128)\oplus k\oplus x_0$.
    Wir nehmen an, dass $b=E(x,k)$, dann gilt
    \begin{align*}
        y_1\oplus b_1&=(r+1\mod 128)\oplus k\oplus x_0
        \oplus (r'+1\mod 128)\oplus k\oplus x_0\\
        &=(r+1\mod 128)\oplus (r'+1\mod 128)
    \end{align*}
    Falls dies stimmt, return 0, andernfalls 1.
\end{document}
