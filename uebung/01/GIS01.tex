\documentclass{article}
\usepackage{tikz}
\usepackage{pgfplots}
\usepackage{amsmath, amsfonts}
\usepackage{enumerate}
\usepackage{hyperref}
\input{title.tex}

\newcommand{\N}{\mathbb{N}}
\newcommand{\R}{\mathbb{R}}
\begin{document}
	\maketitle
	\section{}
	\begin{enumerate}[a)]
	    \item
			$D: K\times Y\rightarrow X$
			\begin{align*}
				(k_1,A)\mapsto a,
				(k_1,B)\mapsto b,
				(k_1,C)\mapsto c,\\
				(k_2,A)\mapsto c,
				(k_2,B)\mapsto a,
				(k_2,C)\mapsto b,\\
				(k_3,A)\mapsto b,
				(k_3,B)\mapsto c,
				(k_3,C)\mapsto a\,
			\end{align*}
		\item
		    No determenistic D exists, because:
			\begin{align*}
				(k_1, A)\mapsto \{a,c\},
				(k_2, B)\mapsto \{a,c\},
				(k_3, C)\mapsto \{a,c\}
			\end{align*}
		\item
    		$D: K\times Y\rightarrow X$
    		\begin{align*}
    			(k_1,A)\mapsto a,
    			(k_1,B)\mapsto b,
    			(k_1,C)\mapsto c,\\
    			(k_2,A)\mapsto b,
    			(k_2,B)\mapsto a,
    			(k_2,C)\mapsto c,\\
    			(k_3,A)\mapsto a,
    			(k_3,B)\mapsto b,
    			(k_3,C)\mapsto c\,
    		\end{align*}
    \end{enumerate}
    \section{}
    \begin{tabular}{c|ccccccccc}
        hex&48&61&6c&6c&6f&77&65&65&6e\\
        \hline
        ASCII&H&a&l&l&o&w&e&e&n
    \end{tabular}

    x is calculated by character: $x(i)=y(i)\oplus k(i)(i\leq 72)$
    \section{}
    \begin{enumerate}[a)]
        \item
            $D:X\times K\rightarrow X$ mit
            $(y,k)\mapsto y(l)y(l)\hdots y(2l-1)\oplus 1$\\
            To show that $S=(X,K,E,D)$ is an encryption scheme,
            we will show $D(E(x,k),k)=x$ for any
            $(x,k)\in X\times K$
            \begin{align*}
                E(x,k)=(r\oplus k)||\left(x\oplus 1^l\right)
            \end{align*}
            Where $|r\oplus k|=l$, thus
            \begin{equation*}
                D(E(x,k),k)=(x\oplus 1^l)\oplus 1^l=x
            \end{equation*}
        \item
            The attacker sends $z_0, z_1$, uses the
            decryption algorithm, which doesn't rely on the
            key and guesses correctly.
    \end{enumerate}
    \section{}
    First we send a text $x$ to encrypt where $x$ fulfills
    all requirements for R-CTR-Vernam.
    Let $y$ be the encrypted text and $k$ the used key,
    $x_0,\hdots, x_n$ the blocks of $x$ of length 128
    and $y_0,\hdots y_n$ the blocks of $y$.

    We now send the 2 texts $z_0=x,z_1$ to be encrypted.
    Let $b_0,\hdots,b_n$ be the blocks of $b$ of length 128.
    and the chunks $b_0,\hdots,b_n$

    Be $r=y_0$ and $r'=b_0$ the randomly generated numbers
    for $y$ and $b$.
    We know, the first encrypted chunk of $x$,
    $y_1=(r+1\mod 128)\oplus k\oplus x_0$,
    and if $b$ is $x$ encrypted, then
    $b_1=(r'+1)\oplus k\oplus x_0$.
    We assume
    \begin{align*}
        y_1\oplus b_1&=(r+1\mod 128)\oplus k\oplus x_0
        \oplus (r'+1\mod 128)\oplus k\oplus x_0\\
        &=(r+1\mod 128)\oplus (r'+1\mod 128)
    \end{align*}
    If this is true, return 0 else 1.
\end{document}
