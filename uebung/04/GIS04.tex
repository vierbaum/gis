\documentclass{article}
\usepackage{tikz}
\usepackage{pgfplots}
\usepackage{amsmath, amsfonts, amssymb}
\usepackage{enumerate}
\usepackage{hyperref}
\usepackage{listings}
\input{title.tex}

\newcommand{\N}{\mathbb{N}}
\newcommand{\Z}{\mathbb{Z}}
\newcommand{\R}{\mathbb{R}}
\begin{document}
    \maketitle
    \section{}
    \begin{enumerate}[(a)]
        \item
            Using $l=2$ we have $x=10000\neq x'=00100$ with$x,x'\in\{0,1\}^5$ and\\
            $h(x')=1-0+0-0+0=1=1-0+0-0+0=h(x)$.
            Thus $h$ is not a collision resistant hash function.
        \item
            be $x,y\in\{0,1\}^{L-1}$ and $x\neq y$.\\
            $H(x||x)=h(x)\oplus h(x)=0=h(y)\oplus h(y)=H(y||y)$.
            Thus $H$ is not a collision resistant hash function.
    \end{enumerate}
    \section{}
    \begin{enumerate}[(a)]
        \item
            for a $x\in\{0,1\}^m$ we construct
            $x'=x\oplus 1^m$ and
            for even $m$
            \begin{align*}
                h(x')&=|x'_0-x'_1\hdots +x'_m|=|-x_0+x_1\hdots-x_m|\\
                &=|x_0-x_1'\hdots+x_m|\\
                &=h(x)
            \end{align*}
            for odd $m$ analogous
            \begin{align*}
                h(x')&=|x'_0-x'_1\hdots -x'_m|=|-x_0+x_1\hdots+x_m|\\
                &=|x_0-x_1'\hdots-x_m|\\
                &=h(x)
            \end{align*}
            thus $h$ is not second-preimage resistant.
        \item
            for a given $v=h(x)$. We have $v'=\text{dec}(v)$ (where dec converts to decimal,
            i.e. dec(110)=6).
            We now construct $x'=(10)^v$, thus
            \begin{equation*}
                h(x')=v\cdot (1-0) = v = h(x)
            \end{equation*}
            thus $h$ is not preimage resistant.
    \end{enumerate}
    \section{}
    We assume $h$ is collision resistant but $h'$ is not.
    We know, there exist $x,x'\in\{0,1\}^*$ and $x\neq x'$
    with $h'(x)=h(x)$. let $\hat{h}:=h'(x)$ and $\hat{h}':=h'(x)$
    with $|\hat{h}|=l+1=|\hat{h}'|$
    \begin{align*}
        h'(x)&=h'(x')\\
        \hat{h}_0
        \hat{h}_1
        \hdots
        \hat{h}_l
        &=
        \hat{h}'_0
        \hat{h}'_1
        \hdots
        \hat{h}'_l\\
        \hat{h}_1
        \hdots
        \hat{h}_l
        &=
        \hat{h}'_1
        \hdots
        \hat{h}'_l\\
        h(x)&=h(x')
    \end{align*}
    which contradicts our assumption, thus $h'$ is collision resistant
    \section{}
    We know that $E_{\text{Vernam}}(x,k)=x\oplus k$.
    Therefore $E_{\text{Vernam}}(0^l,k)=0^l\oplus k=k$.
    We create the following algorithm
    \begin{lstlisting}[mathescape=true];
        k := send($0^l$)
        choose $x\in\{0,1\}^l$ randomly;
        // $E_{\text{Vernam}}(x,k)=T(x,k)$
        t := x xor k;

        V(x, t, k) = Valid $\blacksquare$
    \end{lstlisting}
    The Attacker wins with advantage 1, therefore $M_\text{Vernam}$ is
    insecure.
    \section{}
    \begin{lstlisting}[mathescape=true]
        // n blocks
        choose $x\in\{0,1\}^{n\cdot l}$

        $y_0$ := send($x_0$)
        for i=1..n-1 {
            // Using $x\oplus x \oplus y = y$ we call
            // $E(v\oplus v\oplus y_{i-1}\oplus x_i, k)=E(y_{i-1}\oplus x_i, k)$
            // Which is CBC-MAC for block i
            $y_i$ := send($v\oplus y_{i-1}\oplus x_i$)
        }

        V$(x, y_{n-1}, k)=$valid
    \end{lstlisting}
    \newpage
    \section{}
    \begin{align*}
    c_0=\sum_{j\leq0}(f_1)_j\cdot_F (f_2)_{0-j}&=0\cdot_F 0=0\\
    c_1=\sum_{j\leq1}(f_1)_j\cdot_F (f_2)_{1-j}&=0\cdot_F 1+_F0\cdot_F 0=0\\
    c_2=\sum_{j\leq2}(f_1)_j\cdot_F (f_2)_{2-j}&=0\cdot_F 0+_F0\cdot_F 1+_F1\cdot_F 0=0\\
    c_3=\sum_{j\leq3}(f_1)_j\cdot_F (f_2)_{3-j}&=0\cdot_F 0+_F0\cdot_F 0+_F1\cdot_F 1+_F1\cdot_F 0=1\\
    c_4=\sum_{j\leq4}(f_1)_j\cdot_F (f_2)_{4-j}&=0\cdot_F 1+_F0\cdot_F 0+_F1\cdot_F 0+_F1\cdot_F 1\\
    &+_F0\cdot_F 0=1\\
    c_5=\sum_{j\leq5}(f_1)_j\cdot_F (f_2)_{5-j}&=0\cdot_F 0+_F0\cdot_F 1+_F1\cdot_F 0+_F1\cdot_F 0\\
    &+_F0\cdot_F 1+_F1\cdot_F 0=0\\
    c_6=\sum_{j\leq6}(f_1)_j\cdot_F (f_2)_{6-j}&=0\cdot_F 1+_F0\cdot_F 0+_F1\cdot_F 1+_F1\cdot_F 0\\
    &+_F0\cdot_F 0+_F1\cdot_F 1+_F0\cdot_F 0=0\\
    c_7=\sum_{j\leq7}(f_1)_j\cdot_F (f_2)_{7-j}&=0\cdot_F 0+_F0\cdot_F 1+_F1\cdot_F 0+_F1\cdot_F 1\\
    &+_F0\cdot_F 0+_F1\cdot_F 0+_F0\cdot_F 1+_F0\cdot_F 0=1\\
    c_8=\sum_{j\leq8}(f_1)_j\cdot_F (f_2)_{8-j}&=0\cdot_F 0+_F0\cdot_F 0+_F1\cdot_F 1+_F1\cdot_F 0\\
    &+_F0\cdot_F 1+_F1\cdot_F 0+_F0\cdot_F 0+_F0\cdot_F 1+_F0\cdot_F 0=1\\
    &\Rightarrow f_1\cdot_F f_2=(0,0,0,1,1,0,0,1,1)=x^3+x^4+x^7+x^8
    \end{align*}
\end{document}
